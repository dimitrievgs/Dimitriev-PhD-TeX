% !TEX root = Disser.tex
% !TEX encoding = Windows-1251

\usepackage{cmap}
\usepackage[cp1251]{inputenc}
%\usepackage[T2A,T1]{fontenc} % before babel! %пробуем для аббревиатур
\usepackage[russian]{babel}
%\usepackage[russian, english]{babel} 

%\usepackage[%  %пробуем для аббревиатур
%style=ieee,
%isbn=true,
%url=true,
%defernumbers=true,
%sorting=nyt, % "Sort by name, year, title."
%sorting=none, % "Do not sort at all. All entries are processed in citation order." (order of appearance)
%bibencoding=utf8,
%backend=biber,
%language=auto,    % get main language from babel
%autolang=other,
%]{biblatex}

\usepackage{etoolbox} % Для продвинутой проверки разных условий

%%% Математические пакеты %%%
\usepackage{amsmath}   % Математические дополнения от AMS
\usepackage{amssymb}       % Математические дополнения от AMS

%%% Изображения %%%
\usepackage{graphicx}                  % Подключаем пакет работы с графикой

\usepackage{bm}

\usepackage{color}
\usepackage[dvipsnames]{xcolor} %68 дополнительных цветов

%%% Поля и разметка страницы %%%
\usepackage{pdflscape}                              % Для включения альбомных страниц
\usepackage{geometry}                               % Для последующего задания полей

%%% Гиперссылки %%%
\usepackage{hyperref}

%%% The package setspace lets you easily change the line spacing, and that provides a macro \onehalfspacing
\usepackage{setspace}

\usepackage{enumerate}

\usepackage{gensymb} % для знака градуса
\usepackage{ulem} % зачеркивание текста
\usepackage{array} % для указания размеров столбцов в таблице

\usepackage{tabularx} % для указания столбцов в таблице в процентах
\usepackage{multirow} % таблицы с объединением строк

\usepackage{lastpage}

\usepackage{fancyhdr} % чтобы указать положения номера страницы

%%% Счётчики %%%
\usepackage[figure,table]{totalcount}               % Счётчик рисунков и таблиц
\usepackage{totcount}                               % Пакет создания счётчиков на основе последнего номера подсчитываемого элемента (может требовать дважды компилировать документ)

\usepackage{nameref} % ссылки в тексте