%%% Проверка используемого TeX-движка %%%


%%% Поля и разметка страницы %%%
%\usepackage{pdflscape}                              % Для включения альбомных страниц
%\usepackage{geometry}                               % Для последующего задания полей

%%%% Установки для размера шрифта 14 pt %%%%
%% Формирование переменных и констант для сравнения (один раз для всех подключаемых файлов)%%
%% должно располагаться до вызова пакета fontspec или polyglossia, потому что они сбивают его работу
%\newlength{\curtextsize}
%\newlength{\bigtextsize}
%\setlength{\bigtextsize}{13.9pt}

%%%%%%%%%%%%%%%%%%%%%%%

\usepackage{cmap}
\usepackage[cp1251]{inputenc}
%\usepackage[T2A,T1]{fontenc} % before babel! %пробуем для аббревиатур
\usepackage[russian]{babel}
%\usepackage[russian, english]{babel} 

%\usepackage[%  %пробуем для аббревиатур
%style=ieee,
%isbn=true,
%url=true,
%defernumbers=true,
%sorting=nyt, % "Sort by name, year, title."
%sorting=none, % "Do not sort at all. All entries are processed in citation order." (order of appearance)
%bibencoding=utf8,
%backend=biber,
%language=auto,    % get main language from babel
%autolang=other,
%]{biblatex}

\usepackage{amsmath}
\usepackage{amssymb}
\usepackage{graphicx}
\usepackage{bm}

\usepackage{color}
\usepackage[dvipsnames]{xcolor} %68 дополнительных цветов

%%% Поля и разметка страницы %%%
\usepackage{pdflscape}                              % Для включения альбомных страниц
\usepackage{geometry}                               % Для последующего задания полей

\usepackage{hyperref}
\usepackage{enumerate}

\usepackage{gensymb} % для знака градуса
\usepackage{ulem} % зачеркивание текста
\usepackage{array} % для указания размеров столбцов в таблице

\usepackage{tabularx} % для указания столбцов в таблице в процентах
\usepackage{multirow} % таблицы с объединением строк

\usepackage{lastpage}

%%% Счётчики %%%
\usepackage[figure,table]{totalcount}               % Счётчик рисунков и таблиц
\usepackage{totcount}                               % Пакет создания счётчиков на основе последнего номера подсчитываемого элемента (может требовать дважды компилировать документ)