\graphicspath{{Images/}}

\newtotcounter{citenum}
\def\oldcite{}
\let\oldcite=\bibcite
\def\bibcite{\stepcounter{citenum}\oldcite}

\newcommand{\com}[1]{} %for multiline comments

%\usepackage{subfigure}

%%% 
%\newcommand{\erf}{\mathop{\rm erf}\nolimits}
%\newcommand{\arch}{\mathop{\rm Arch}\nolimits}
%\newcommand{\floor}{\mathop{\rm floor}\nolimits}
%\newcommand{\divv}{\mathop{\rm div}\nolimits}
%\newcommand{\grad}{\mathop{\rm grad}\nolimits}
%\newcommand{\rot}{\mathop{\rm rot}\nolimits}
%\newcommand{\sign}{\mathop{\rm sign}\nolimits}
%\newcommand{\res}{\mathop{\rm res}}
%\newcommand{\addMisha}[1]{#1}
%%%

\newcommand{\red}[1]{{\color{red}#1}}

\newcommand{\av}[1]{\left\langle #1\right\rangle}
\newcommand{\rmi}{{\rm i}}
\newcommand{\rmd}{{\rm d}}
\newcommand {\e}{{\rm e}}
\newcommand {\Tr}{{\rm Tr}\,}
\renewcommand {\i}{{\rm i}}
\renewcommand {\Re}{\mathop{\mathrm{Re}}\nolimits}
\renewcommand {\Im}{\mathop{\mathrm{Im}}\nolimits}

\renewcommand{\eqref}[1]{(\ref{#1})}




%% \newcommand{\red}[1]{{\color{red} #1}}
%\newcommand{\red}[1]{#1}
%\newcommand{\A}{{\cal A}}
%\newcommand{\B}{{\cal B}}
%\renewcommand{\C}{{\cal C}}
%\newcommand{\D}{{\cal D}}
%\newcommand{\F}{{\cal F}}
%\newcommand{\M}{{\cal M}}
%\newcommand{\N}{{\cal N}}
%\newcommand{\TM}{{\cal TM}}
%\newcommand{\braket}[3]{\left\langle #1 \left| #2 \right| #3 \right\rangle}
%\newcommand{\ket}[1]{\left| #1 \right\rangle}
%\newcommand{\MT}{\widetilde{\cal TM}}
%\newcommand{\eps}{\varepsilon}
%\newcommand{\dd}[1]{\frac{\partial^2}{\partial #1^2}}
%\newcommand{\dz}{\frac{\partial}{\partial z}}
%\newcommand{\dx}{\frac{\partial}{\partial x}}
%\newcommand{\dy}{\frac{\partial}{\partial y}}
%\newcommand{\dphi}{\frac{\partial}{\partial \varphi}}
%
%
%\renewcommand{\Im}{\mathop{\mathrm{Im}}\nolimits}
%\renewcommand{\Re}{\mathop{\mathrm{Re}}\nolimits}
%
%\newcommand{\arctanh}{\mathop\mathrm {arctanh}\nolimits}
%\newcommand{\e}{{\rm e}}
%\newcommand{\rmi}{{\rm i}}

\def\dfrac{\displaystyle\frac}
\renewcommand\chaptername{Глава}
\renewcommand\figurename{Рис.}
\renewcommand\tablename{Таблица}
\renewcommand\contentsname{Содержание}
\renewcommand\bibname{Список литературы}

%%% Интервалы %%%
%% По ГОСТ Р 7.0.11-2011, пункту 5.3.6 требуется полуторный интервал
%% Реализация средствами класса (на основе setspace) ближе к типографской классике.
%% И правит сразу и в таблицах (если со звёздочкой)
%\DoubleSpacing*     % Двойной интервал
\onehalfspacing    % Полуторный интервал
%\setSpacing{1.42}   % Полуторный интервал, подобный Ворду (возможно, стоит включать вместе с предыдущей строкой)

%%% Макет страницы %%%
% Выставляем значения полей (ГОСТ 7.0.11-2011, 5.3.7)
\geometry{a4paper, top=2cm, bottom=2cm, left=2.5cm, right=1cm, nofoot, nomarginpar} %, heightrounded, showframe
\setlength{\topskip}{0pt}   %размер дополнительного верхнего поля
\setlength{\footskip}{12.3pt} % снимет warning, согласно https://tex.stackexchange.com/a/334346

%%% Выравнивание и переносы %%%
%% http://tex.stackexchange.com/questions/241343/what-is-the-meaning-of-fussy-sloppy-emergencystretch-tolerance-hbadness
%% http://www.latex-community.org/forum/viewtopic.php?p=70342#p70342
\tolerance 1414
\hbadness 1414
\emergencystretch 1.5em % В случае проблем регулировать в первую очередь
\hfuzz 0.3pt
\vfuzz \hfuzz
%\raggedbottom
%\sloppy                 % Избавляемся от переполнений
\clubpenalty=10000      % Запрещаем разрыв страницы после первой строки абзаца
\widowpenalty=10000     % Запрещаем разрыв страницы после последней строки абзаца
\brokenpenalty=4991     % Ограничение на разрыв страницы, если строка заканчивается переносом

%%\renewcommand\baselinestretch{1}
%\hbadness 10000 \tolerance 10000 \pagestyle{empty} \textwidth
%15.5cm \oddsidemargin 1.0cm \textheight 22cm \topmargin -1cm

%   Parameters for FLOAT pages (not text pages):
\renewcommand{\floatpagefraction}{0.6}  % require fuller float pages
% N.B.: floatpagefraction MUST be less than topfraction !!
\renewcommand{\dblfloatpagefraction}{0.7}   % require fuller float pages

% Different font in captions
\newcommand{\captionfonts}{\small}

\makeatletter  % Allow the use of @ in command 
\long\def\@makecaption#1#2{%
  \vskip\abovecaptionskip
  \sbox\@tempboxa{{\captionfonts #1: #2}}%
  \ifdim \wd\@tempboxa >\hsize
    {\captionfonts #1: #2\par}
  \else
    \hbox to\hsize{\hfil\box\@tempboxa\hfil}%
  \fi
  \vskip\belowcaptionskip}
\makeatother   % Cancel the effect of \makeatletter


\binoppenalty=10000 \relpenalty=10000